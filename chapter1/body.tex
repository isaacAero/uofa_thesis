% \published{statement_of_authorship.jpg}
\chapter{ Description of the model \label{ch:numero_uno}}
\section{Modeling Growing Geometry }
Cell colonies are modeled using level-sections of implicit curves in two dimensions. For example, a circle can be modeled as the level-0 set of the following equation
\begin{equation*}
    f(x,y) =R^2-( x^2 + y^2),
\end{equation*}
where $R$ is the radius of the circle. If the filled in circle, the disk, is required, then producing a surf plot of the function in the region where $f(x,y) \geq 0$ is sufficient. This can be achieved by modifying the value of $f(x,y)$ to be \codeword{nan} wherever $f(x,y)<0$ so that MATLAB's surf function does not plot it.
\\
\\
For approximately circular cells, each individual cell is modeled via the equation of a circle centered at position $(x_j,y_j)$ where $j$ indexes over the current total number of cells. The equation for a single cell is therefore given by
\begin{equation*}
    f_j(x,y) =R_j^2- \left[ (x-x_j)^2 + (y-y_j)^2\right],
\end{equation*}
where $R_j$ is the radius of the $j$-th cell. We will discuss how multiple cells can be blended together in one colony level implicit curve. Given the implicit curve for cell $1$ and cell $2$, respectively $f_1(x,y)$ and $f_2(x,y)$, we blend them with the following transformation
\begin{equation*}
    \Phi(x,y) = \ln{ \left[ \frac{1}{2 k} \sum_{j=1}^N{ e^{k f_j(x,y)}} \right]},
\end{equation*}
where $\Phi$ is related to the colony microscopic density.








