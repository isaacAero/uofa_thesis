\introduction
\epigraph{Pure mathematics can be practically useful and applied mathematics can be artistically elegant}{\textit{Paul Halmos}}
At the beginning, we imagine a colony of yeast cells that is restricted to move along a straight line. The colony is therefore modeled as a set of real numbers $C \subset \mathbb{R}$. Like an archipelago of small islands, the colony as a whole is composed of closed sets, $C_j$ where $j \in \mathbb{N}$ indexes over the cells. Therefore, the colony is a disjoint union of these individual cells,
\begin{equation*}
    C = \coprod_{j=1}^N C_j
\end{equation*}
and $N$ is the total cell count. Closed sets (in the standard topology on $\mathbb{R}$) are chosen to represent the cells for the technical reason that a point (singleton) may also constitute a cell. In fact, we further restrict each cell to an open interval $C_j = [a_j, b_j]$.
\\
\\
As we shall see in general, all we require is that each cell have no holes. Another way of saying this is that each cell is homeomorphic to a singleton (which works in $\mathbb{R}^2$ and $\mathbb{R}^3$ as well). Finally, the closed interval $[a_j,b_j]$ will be called a parametrization of the cell $C_j$: this is important to note for the generalization to $\mathbb{R}^2$ and $\mathbb{R}^3$, where parametrizations must also be constructed.
\\
\\
The mechanism of mitosis must account for the fact that several cells can undergo fission at the same time, or more aptly, they undergo mitosis independently. The most general formulation, which is also simple to implement computationally is the addition of a non-intersecting singleton $\{x  \}$ to the set $C$. Note, that this mechanism is chosen principally for the fact that it conserves the Lebesgue measure of the colony,
\begin{equation*}
    l(C \cup \{x\} ) = l(C),
\end{equation*}
where $C \cup \{x\} $ is the colony after mitosis has occurred, since singletons have measure $0$.