\chapter{Abstract}
Baker's yeast (\textit{Saccharomyces cerevisiae}) is capable 
of undergoing a pseudohyphal transition to filamentous growth in nutrient poor environments.
An algebriac representation of \textit{S. cerevisiae} cells is adapted from
computer graphics to model cellular division (mitosis) at low computational cost.
By reimagining mitosis as topological bifurcation using level sections 
of signed distance functions (SDFs) underpinned by a discrete mechanistic network, 
novel predictions about 
\textit{S. cerevisiae} colony morphology are made. Coupling 
the algbraic biomass field to a traditional reaction diffusion system 
for nutrient allows the relationship between metabolism 
and growth rate to be estimated for this agent based 
model (ABM). A measure for filamentous branching called compactness 
is introduced and related to cellular mobility under chemotaxis through
numerical experiments.  
