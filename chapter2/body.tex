
\chapter{Applying the model}
\section{Non-dimensionalising the equations of motion}
When the number of cells is fixed at $N \geq 1$ we have the following equations of motion,
which consist of the equations of motion for the biomass node network (governed by forces) and 
the reaction diffusion PDE of the nutrient field. Let the spring connectivity
be given by the matrix $A_{ij}$ where $i$ and $j$ index over the number of cells and $A_{ij}$
is $1$ if node $i$ is connected to node $j$ by an edge and $0$ otherwise.
\begin{equation*}
    \frac{d \vb{x}_i }{dt} = 
    \frac{1}{\eta} \left[\sum_{i = 1, i \neq j}^N   \left(A_{ij} \vb{F}_{\textrm{spring},ij} + 
     \vb{F}_{\textrm{contact},ij} \right) +\vb{F}_{\textrm{chemo},i}  \right] ,
    \ \textrm{for} \ i = 1, ... , N.
\end{equation*}
\begin{equation*}
    \pdv{c}{t} = D \left(\pdv[2]{c}{x} + \pdv[2]{c}{y} \right) - r c b,
\end{equation*}
where the biomass field $b$ is given by 
\begin{equation*}
b(x,y,t) = \begin{cases}
            -\beta g(x,y,t), & \ \textrm{if} \ g(x,y,t) \leq 0, \\
                0, &    \ \textrm{otherwise},
           \end{cases}
\end{equation*}
where the field $g(x,y,t) =\textrm{smoothmin}(f_1(x,y,t), ...,f_N(x,y,t) )$ where 
$f_i(x,y,t)$ is the signed distance field for the $i$-th cell.
Here the spring force is given by 
\begin{equation}
    \vb{F}_{\textrm{spring},ij} = -\left( ||\vb{x}_i - \vb{x}_j|| - L_0\right) \frac{\vb{x}_i - \vb{x}_j}{||\vb{x}_i - \vb{x}_j||},
\end{equation}
the contact force is given by 
\begin{equation}
    \vb{F}_{\textrm{contact},ij} = F H(R - ||\vb{x}_i - \vb{x}_j||) \frac{\vb{x}_i - \vb{x}_j}{||\vb{x}_i - \vb{x}_j||} ,
\end{equation}
and the force of chemotaxis is given by 
\begin{equation*}
    \vb{F}_{\textrm{chemo},i} = \gamma \nabla c (\vb{x}_i, t),
\end{equation*}

\section{From qualitative to quantitative}
The model developed thus far is a numerical framwork for simulating growing cell colonies.
There is an element of randomness in the model: when two nodes are dislodged from the 
same point there is initially no preferred direction to move in so one must be chosen randomly
before other forces can take effect. For this reason, every seperate run of the model 
starting with identical initial conditions will look very different after time has passed. 
It is expected however that some ``averaged''
quantity will stabilise so long as the model is simulated over a fairly large number of runs.
Seperate runs of the model belong to a set called an ensemble.
We will start with a relatively straightforward metric, the number of cells at time
$T = 500$ steps. 











