
\chapter{Software Implementation: \textit{CellColonySimulator}}


\begin{figure}[!htb]
    \centering


    \begin{tikzpicture}[every text node part/.style={align=center}, 
                    node distance=2cm]


    \begin{scope}[scale=0.85,transform shape]
    \node (start) [startstop] {\codeword{Master.m}};
    \node (in1) [io, below of=start] { Model parameters: $\lambda_1, ... , \lambda_7$ \\
                                       Numerical parameters: $h$, $\Delta t$ \\
                                       Run parameters: \codeword{ensembleSize}, \codeword{sampleTimes}};
    \node (pro1) [process, below of=in1] {\codeword{runSimulation.m}};
    \node (pro2) [process, below of=pro1] {\codeword{ellipse.m}};
    \node (pro3) [process, below of=pro2] {\codeword{updateFood.m}};
    \node (dec1) [decision, below of=pro3, yshift=-1.5cm] {Mitosis Condition};
    \node (pro4) [process, below of=dec1, yshift=-1.5cm] {\codeword{addCell.m}};
    \node (proNothing) [process, left of=pro4, xshift=-1.5cm] {Proceed};
    \node (pro5) [process, below of=pro4] {\codeword{updateNodes.m}};
    %\node (pro5) [process, right of=dec1, xshift=2cm] {Process 2b};
    \node (out1) [io, below of=pro5] {Run Statistics at \codeword{sampleTimes}: \\
                                      \codeword{compactness},
                                      \codeword{averageRadius},
                                      \codeword{cellCount},
                                      $\bar{\mu}$};
    \node (stop) [startstop, below of=out1] {Plot ensemble averaged results};
    \node (for) [process, left of=pro3, xshift=-3.5cm] {Loop over ensemble};
    \node (forTime) [process, right of=pro3,  xshift=3.0cm] {Loop over time};

    \draw [arrow] (start) -- (in1);
    \draw [arrow] (in1) -- (pro1);
    \draw [arrow] (pro1) -- (pro2);
    \draw [arrow] (pro2) -- (pro3);
    \draw [arrow] (pro3) -- (dec1);
    \draw [arrow] (dec1) -| node[anchor=south]{else} (proNothing);
    \draw [arrow] (proNothing) |- node[anchor=south]{} (pro5);
    \draw [arrow] (dec1) -- node[anchor=east]{if}(pro4);
    \draw [arrow] (pro4) -- (pro5);
    \draw [arrow] (pro5) -- (out1);
    \draw [arrow] (out1) -- (stop);
    \draw [arrow] (in1) -| node[anchor=south]{} (for);
    \draw [arrow] (for)  |- node[anchor=south]{} (out1);
    \draw [arrow] (pro1)  -| node[anchor=south]{} (forTime);
    \draw [arrow] (forTime)  |- node[anchor=south]{} (pro5);
    
    %\draw [arrow] (pro2a) -- (out1);
    %\draw [arrow] (out1) -- (stop);

    \end{scope}
\end{tikzpicture}
\caption{A flow chart of \textbf{CellColonySimulator} software package in MATLAB. }
\label{fig:softwareFlowChart}
\end{figure}

\section{The structure of the program}
The program \textbf{CellColonySimulator} is implemented \textit{in-house} with MATLAB as shown in 
figure \ref{fig:softwareFlowChart}. \textbf{CellColonySimulator} is broken up 
into six seperate scripts as per the programming principle of modularity. These include:
\codeword{Master.m},
\codeword{runSimulation.m},
\codeword{ellipse.m},
\codeword{updateFood.m},
\codeword{addCell.m},
\codeword{updateNodes.m}.

\subsection{Calling the update loop in \textit{runSimulation.m}}
The function \codeword{runSimulation.m} takes in the model parameter vector,
$\lambda$, the struct of numerical discretisation parameters, \codeword{numericPack},
the times at which statistics are sampled, \codeword{sampleTimes}, 
the colony data including node positions $(x,y)$ and the network adjacency
matrix, \codeword{connectivity}, and the initial nutrient field 
given by \codeword{food}.
\\

The mesh grid matrices $X$ and $Y$ are taken from \codeword{numericPack} and 
converted to graphics processing unit (GPU) arrays using MATLAB's \codeword{gpuArray()}. This is crucial 
for speed as GPU operations on large matrices are highly optimised and parallelised within 
the GPU architecture. Besides generating descriptive file names for the output plots
and other miscellaneous tasks, \codeword{runSimulation.m} then progresses into 
the primary \codeword{for} loop over the time indices, \codeword{timeIndex}.
\\

A task to print to the console to let the user know what ensemble instance
and \codeword{timeIndex} the program is up to, is done first.
Secondly, based on the current node positions $(x_i,y_i)$ for
$i \in \{1, ..., N_{\textrm{nodes}}\}$, and the number
of edges present in the \codeword{connectivity} matrix, the current number of cells
$N_{\textrm{cells}}$ is counted up
and vectors are associated to the variables $x_{c,k}$,$y_{c,k}$,
$\theta_k$, $p_k$ and $q_k$ (based on equations
 \ref{eqn:semiMajorAxis}, \ref{eqn:semiMinorAxis},
\ref{eqn:orientationAngle} and \ref{eqn:centerPos}) where $k \in \{1, ..., N_{\textrm{cells}}\}$.
These vectors are passed to the function \codeword{ellipse()} (explained in 
subsection \ref{ssec:ellipse})
which computes the colony-level signed distance field (SDF),
which is comprised of the elliptical SDF of each cell indexed $k$.
\\

At this stage the biomass field, \codeword{biomass}, is computed 
from the colony level SDF, based on equation \ref{eqn:biomass}.
The current value of the nutrient field and the biomass field,
are passed to \codeword{updateFood()} (explained in 
subsection \ref{ssec:updateFood}) which updates the nutrient field
based on the Crank-Nicholson method applied to equation \ref{eqn:EOMs_PDE}.
The food at the next time step is the output from \codeword{updateFood()}.
\\

The value of the growth rate across the colony $\mu_i$ is then determined
using 2D interpolation by calling \codeword{interp2(X, Y, food, x, y)},
where $X, Y$ are the underlying grid matrices, \codeword{food} is the updated nutrient 
field, and $(x,y)$ are the node positions, taken as query points in the interpolation.
Recall from equation \ref{eqn:growthRate}, the growth rate 
is given by the the average of the nutrient field sampled over the nodes.
\\

The updated number of nodes, $N_{\textrm{nodes}}' = N_{\textrm{nodes}}(t_{n + \Delta n})$, is found 
by multiplying the current number of nodes, $N_{\textrm{nodes}}$,
by $e^{\mu}$ (as stipulated by the exponential growth in equation \ref{eqn:expGrowth}) where $\mu$ is the average growth rate over the colony 
as discussed above. The number of nodes added is then given by 
\begin{equation*}
    \Delta N_{\textrm{nodes}} = N_{\textrm{nodes}}' - N_{\textrm{nodes}} = \lceil e^{\mu} -1 \rceil N_{\textrm{nodes}},
\end{equation*}
where $\mu$ is given by 
\begin{equation*}
    \mu = \frac{1}{N_{\textrm{nodes}}} \sum_{i = 1}^{N_{\textrm{nodes}}} \mu_i,
\end{equation*}
where $\mu_i = c(\vb{x}_i, t)$ is the value of the nutrient field $c(\vb{x}_i, t)$ sampled 
at the $i$-th node position (this is just a recapitulation of equation \ref{eqn:growthRate}).
The number of time steps per mitosis event, $\Delta n$ is given by \codeword{ceil(1.0/dt)},
and we use an \codeword{if} statement to ensure \codeword{addCell()} is only called when $n$ 
is a multiple of $\Delta n$.
\\

In \codeword{addCell()}, the number of nodes to add, $\Delta N_{\textrm{nodes}}$, is passed 
as a argument. The mechanism to choose the parent cells to undergo mitosis is 
explained in subsection \ref{ssec:addCell}. Finally, 
the new node positions (based on equation \ref{eqn:EOMs_ODE}) are determined 
using \codeword{updateNodes()} as detailed in subsection \ref{ssec:updatenodes}.
\\

The rest of \codeword{runSimulation.m} is devoted to plotting the biomass and nutrient 
fields, aswell as generating the underlying node network using MATLAB's 
\codeword{graph()} where the colony adjacency matrix, \codeword{connectivity},
is passed as an argument.


\subsection{Vectorised computation of colony SDF in \textit{ellipse.m}}\label{ssec:ellipse}
The custom function \codeword{ellipse.m} has been iterated
with high efficiency in mind since \codeword{ellipse.m} has to be called 
every time step. Using MATLAB's widget \textit{run-and-time}, the bottleneck found
in \codeword{ellipse.m} was eliminated. This significantly 
improved run times. How was this speedup achieved?
\\

Where previously \codeword{ellipse.m} required a \codeword{for} loop over the SDFs for 
each cell, the final iteration of \codeword{ellipse.m} uses 
completely vectorised operations which run fast on the GPU. I have provided 
the code below.



\begin{lstlisting}[style=Matlab-editor,  
                   caption={A code listing for the \textbf{ellipse.m} script},captionpos=b]
function sdf =                ... %Colony level SDF
    ellipse(X, Y,             ... %Grid values
            theta,            ... %Orientation angles
            centerX, centerY, ... %Center coordinates
            semiMajorRadius,  ... %p_k
            semiMinorRadius   ... %q_k       
           ) 

    %Translate the field to the center coordinates:
    Xt = repmat(X, [1,1,size(centerX,2)])-reshape(centerX, [1,1,size(centerX,2)]);
    Yt = repmat(Y, [1,1,size(centerX,2)])-reshape(centerY, [1,1,size(centerX,2)]);


    %Precompute trigonometric functions for efficiency
    COS = cos(theta);
    SIN = sin(theta);
    cosReshape = reshape(COS, [1,1,size(centerY,2)]);
    sinReshape = reshape(SIN, [1,1,size(centerY,2)]);

    %Rotate translated field by the rotation matrix (correctly vectorised):
    rotX =  Xt .* cosReshape + Yt .* sinReshape; 
    rotY = -Xt .* sinReshape + Yt .* cosReshape;

    %Precompute reciprocal squared of geometry arrays:
    semiMajorRadiusReciprocalReshape2 = ... 
        reshape(1.0 ./ semiMajorRadius.^2, [1,1,size(centerX,2)]);
    semiMinorRadiusReciprocalReshape2 = ...
        reshape(1.0 ./ semiMinorRadius.^2, [1,1,size(centerX,2)]);

    %Instead of using smoothmin (slow) just use min of the
    %square of the SDF. [This does not require calling sqrt() 
    %which is computationally very costly]
    sdf = min(rotX .^2 .* semiMajorRadiusReciprocalReshape2 + ...
              rotY .^2 .* semiMinorRadiusReciprocalReshape2 - ... 
              1.0, [], 3);

    %Normalise to with magnitude 1 in colony region:
    absMin = abs(min(sdf,[], "all"));
    sdf = sdf ./ absMin ;

end    
\end{lstlisting}

One thing to note from the MATLAB code listing is that 
vectorised division, say called with \codeword{./}, 
is precomputed as reciprocals which 
are then multiplied through in the compuatation of \codeword{sdf} using \codeword{.*} which is 
faster. Precomputing trigonometric functions was also opted for with 
the aim of reducing repeated compuatations with the same angle. 
Finally, we recall the discussion in section \ref{sec:introSDFs}, in which 
it was noted that \codeword{min} is significantly faster than evaluating smoothmin. 
The final optimisation made was to remove the call to \codeword{sqrt()} which  
actually has no effect on the shape of each cell but vastly improves
compute time.


\subsection{A fast Crank-Nicolson scheme in \textit{updateFood.m}}\label{ssec:updateFood}
The Crank-Nicolson scheme is an implicit numerical scheme that requires the 
computation of a (mostly zero) matrix of coefficients \codeword{LHM} (which has 
dimensions $N_{\textrm{grid}}^2 \times N_{\textrm{grid}}^2$)
a right hand side column vector \codeword{RHS} ($N_{\textrm{grid}}^2 \times 1$), and the 
solution of the linear system 
\codeword{LHM * x = RHS}. 
\\

The most obvious optimisation made in \codeword{updateFood()} 
is to completely vectorise all operations. As evinced in the code 
listing supplied below, there are no \codeword{for} loops.
\begin{lstlisting}[style=Matlab-editor,basicstyle=\lstconsolas,basicstyle=\small,  
                   caption={A code listing for the \textbf{updateFood.m} script},captionpos=b]
function foodNew = updateFood(biomass, food, lambda, dt, h, X)

    N = size(X,1);
    V = N .* N;
    Vi = (N-2) .* (N-2);

    i = 2:(N-1);
    j = i;

    
    %Generate the five point stencil of indices in row major order.
    ij   = rmo(i,j,N);
    im1j = rmo(i-1,j,N);
    ip1j = rmo(i+1,j,N);
    ijm1 = rmo(i,j-1,N);
    ijp1 = rmo(i,j+1,N);

    %Flatten these:
    fij   = ij(:);
    fim1j = im1j(:);
    fip1j = ip1j(:);
    fijm1 = ijm1(:);
    fijp1 = ijp1(:);


    CFL = dt .* lambda(1) ./ (h .* h);
    coeff1 = 0.5 .* dt .* biomass(i,j) + 2.0 .* CFL + 1.0;
    coeff2 = 0.5 .* CFL;
    coeff3 = 1.0 - 2.0 .* CFL - 0.5 .* dt .* biomass(i,j);

    %Flatten the coefficients:
    coeff1 = coeff1(:);
    coeff3 = coeff3(:);

   

    c = food(:);

    RHS = coeff3 .* c(fij) + ...
          coeff2 .* (c(fim1j) + c(fip1j) + c(fijm1) + c(fijp1));

    


    LHM = spdiags([-coeff2 .* ones(Vi,1) ,-coeff2 .* ones(Vi,1) ,...
                   gather(coeff1),...
                   -coeff2 .* ones(Vi,1), -coeff2 .* ones(Vi,1)],...
                  [-N, -1, 0, 1, N], Vi, Vi);
    yM  = spdiags(-coeff2 .* ones(1,N), 0, N, N);


    LHM(1:N,(end - N + 1):end) = LHM(1:N,(end - N + 1):end) + yM;
    LHM((end - N + 1):end,1:N) = LHM((end - N + 1):end,1:N) + yM;


    [cNew, flag] = pcg(LHM,RHS,1e-6);

    foodNew = food;

    foodNew(i,j) = reshape(cNew, [N-2, N-2]);

end


function index = rmo(i,j,N)
    index = i' + N .* (j-1);
end
\end{lstlisting}

Another optimisation which was not only efficient but also necessary to avoid
memory issues, was to construct \codeword{LHM} with sparse matrices, which 
meant that on the order of $N_{\textrm{grid}}^4$ redundant zeros did not have 
to be stored in memory. In particular, the pentadiagonal matrix ($5$ non-zero diagonals)
 \codeword{LHM} was constructed using \codeword{spdiags}. A crucial optimisation 
from the point of view of perfomance was to 
replace the direct linear solution, \codeword{LMH \ RHS}, with the function, 
\codeword{pcg}, which carries out the preconditioned conjugate 
gradient numerical method at no appreciable change in accuracy (error bounded by \codeword{1e-6}),
but significantly faster.
\\

Regarding indexing, note that MATLAB uses column-major-order by default in 
its vectorised functions (see MathWorks website). To align with this convention,
the row-major-order function, \codeword{rmo}, transposes the row indices, \codeword{i},
and leaves the column indices, \codeword{j}, fixed. Other configurations were 
trialed but this was ultimately the only one that produced the correct 
nutrient diffusion dynamics.



\subsection{Introducing new nodes in \textit{addCell.m}} \label{ssec:addCell}

\subsection{Advancing the node positions using Euler's method in \textit{updateNodes.m}} \label{ssec:updatenodes}








