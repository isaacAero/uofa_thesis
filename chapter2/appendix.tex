\chapter{My appendices from chapter 1 \label{app:chapter_1}}
% I like to do a seperate appendix file for each chapter and save it in the appropriate folder

\section{First appendix section}
\section{Cell-cell collisions with constrained dynamics}
Earlier, it was mentioned that the intersection of cells could be found by taking a smoothmax. To find the overlapping area
the sum of the grid points in the intersection can be taken and then multiplied by the grid square area $h_x h_y$. Whilst
it is more or less trivial to calculate the overlapping area, ensuring that this area remains zero throughout the simulation
is more involved. We employ the technique explained by \cite{witkin1997introduction} based on constrained
dynamics. The constraint in this case is that the area $C$ remains $0$ for all times,
\begin{equation}
    C(\vb{q}) = 0,
\end{equation}
where $\vb{q}$ is the concatenated state vector of the system. One should be aware that, if we have multiple colonies, it will
be required to have pairwise constraints forcing no overlap between each of the constitiuent colonies. For the purpose of simplicity
our state vector $\vb{q}$ will just take into account position and orientation and not size of cells. Later on, this will be generalized.
Therefore $\vb{q}$ is produced by concatenating over $\vb{q}_j$ into one $3N \times 1$ column vector, where
\begin{equation}
\vb{q}_j = \begin{bmatrix}
                x_j \\
                y_j \\
                \theta_j
            \end{bmatrix}.
\end{equation}
Of course, the number of variables per cell can be larger but this comes at a cost in computational time. Recall that the 
intersecting area is secretly a function of the cell state coordinates $\vb{q}_j$ since we are taking a smoothmax of the SDFs 
of each cell. In other words,
\begin{equation*}
f_{\textrm{intersect}}(x,y,\vb{q}) = \textrm{smoothmax}(f_1(x,y,\vb{q}) , \ldots, f_N(x,y,\vb{q}) ),
\end{equation*}

for concreteness, we write out the formula for smoothmax which is the negative smoothmin of the negatives of the SDFs or
\begin{equation*}
    f_{\textrm{intersect}}(x,y,\vb{q}) = -\textrm{smoothmin}(-f_1(x,y,\vb{q}) , \ldots, -f_N(x,y,\vb{q}) ).
\end{equation*}
In full this boils down to,
\begin{equation*}
    f_{\textrm{intersect}}(x,y,\vb{q}) = k \log( \sum_{j=1}^N e^{f_j(x,y,\vb{q})/k}).
\end{equation*}
Now we assume the state is given by five variables for full generality:
\begin{equation}
\vb{q}_j(t) = 
\begin{bmatrix}
    x_j (t) \\
    y_j(t) \\
    \theta_j(t) \\
    b_j(t) \\
    R_j(t)
\end{bmatrix},
\end{equation}
where $b(t)$ is the semi-major axis length and $R_j(t) \in (0,1]$ is the aspect ratio of the elliptical cell so the semi-minor axis is given by $a_j(t) = R_j(t) b_j(t)$. 
Why are we going to the effort to write out $C(\vb{q})$ in full? Because we need to compute the Jacobian of $C(\vb{q})$ with
respect to $\vb{q}$ in order to carry out the constrained dynamics algorithm. The upshot is that smoothmax is differentiable
whereas maximum is not differentiable. 

We define the region in $\mathbb{R}^2$ to integrate over by 
\begin{equation}
    \Omega(\vb{q}) = \{ (x,y) \in \mathbb{R}^2 \ | \ f_{\textrm{intersect}}(x,y,\vb{q}) \leq 0 \},
\end{equation}
so that the area of the overlapping region is simply a two-dimensional integral over $\Omega(\vb{q})$ of $1$,
\begin{equation}
    C(\vb{q}) = \int\int_{\Omega(\vb{q})} dxdy.
\end{equation}
Now we break up the integral into a sum over simply connected components (SCC) so that we can safely apply Leibniz's rule
\begin{equation}
    C(\vb{q}) = \sum_{i \in SCC(\vb{q})}\int\int_{\Omega_i(\vb{q})} dxdy.
\end{equation}
Leibniz's rule tells us how to carry out a total derivative of $\int\int_{\Omega_i(\vb{q})} dxdy$ with respect to $t$ which determines $\vb{q}$. Let's take that total
derivative now to attain
\begin{equation}
   \frac{d}{dt}\int\int_{\Omega_i(\vb{q})} dxdy = \int\int_{\Omega_i(\vb{q})} \pdv{(1)}{t} dxdy + 
   \int_{\partial \Omega_i(\vb{q})} (1) \vb{v}_{\partial \Omega_i(\vb{q})} \cdot \hat{\vb{n}} dl,
\end{equation}
where $\vb{v}_{\partial \Omega_i(\vb{q})}$ is the ``Eulerian'' velocity of the boundary of the $i$-th SCC. All of this can be avoided if we integrate over a smoothstep
which is defined in our region.
\begin{equation}
    C(\vb{q}) = \int\int_{D} \left[ \frac{1}{2} +\frac{1}{2} \tanh{ \left(- \frac{1}{K}f_{\textrm{intersect}}(x,y,\vb{q}) \right)}\right]dxdy,
\end{equation}
where $D$ is the entire domain of the pertri dish.


\begin{equation*}\
\pdv{f_{\textrm{intersect}}(x,y,\vb{q})}{\vb{q}} = \frac{ \sum_{j=1}^N \pdv{f_j(x,y,\vb{q})}{\vb{q}}e^{f_j(x,y,\vb{q})/k}}{ \sum_{j=1}^N e^{f_j(x,y,\vb{q})/k}}.
\end{equation*}
Conveniently, computing the Jacobian, has turned into $N$ problems that are easier to solve individually. Namely computing $\pdv{f_j(\vb{q})}{\vb{q}}$.
This requires us to express the SDF for an ellipse in terms of the query point $(x,y)$, the center of the cell $(x_j,y_j)$, and its orientation $\theta_j$. This
is given in terms of $l_j^-(\vb{q}, x, y)$ and $l_j^+(\vb{q}, x, y)$ which are given in the ellipse formula. We use MATLAB's symbolic toolbox to compute the
ellipse Jacobian $\pdv{f_j(x,y,\vb{q})}{\vb{q}}$ and return a function that can take numerical inputs using \codeword{matlabFunction(J,"File","ellipseJacobian")}.
With that we can compute Jacobian of the constraint using
\begin{equation} 
    \pdv{C(\vb{q})}{\vb{q}} = -\frac{1}{2K} \int \int_D \left[ 1- \tanh[2]( -\frac{f_{\textrm{intersect}}(x,y,\vb{q})}{K}) \right] \pdv{f_{\textrm{intersect}}(x,y,\vb{q})}{\vb{q}} dx dy.
\end{equation}
Now that all the derivatives have been computed analytically, we can carry out the computation of the integral over $x$ and $y$ using a numerical integral in MATLAB.
From now we use $f$ instead of $f_{\textrm{intersect}}$ for brevity, and we pass to index notation, instead expression the $\beta$-th component of $J$ as 
\begin{equation}
J_{\beta} = -\frac{1}{2K} \int \int_D \left[ 1- \tanh[2]( -\frac{f(x,y,\vb{q})}{K}) \right] \pdv{f(x,y,\vb{q})}{q_{\beta}} dx dy.
\end{equation}
For the computation of constraint forces, we need to further compute the total derivative of the Jacobian with respect to time. This turns 
out to be realted to the Hessian matrix of $C(\vb{q})$ (for no explicit time dependence), as 
\begin{equation*}
    \dot{J}_{\beta} = \sum_{\alpha = 1}^{5N} \dot{q}_{\alpha} \frac{\partial^2 C}{ \partial q_{\alpha} \partial q_{\beta}} = \sum_{\alpha = 1}^{5N} \dot{q}_{\alpha} H_{\alpha, \beta},
\end{equation*}
We need to compute how the Hessian of $C$ can be written in terms of the Hessian and Jacobian of $f$.
\begin{equation}
    \frac{\partial^2 C}{ \partial q_{\alpha} \partial q_{\beta}} = \frac{\partial }{\partial q_{\alpha}} J_{\beta}
\end{equation}
Crunching the calculuations we get
\begin{equation}
 \frac{\partial }{\partial q_{\alpha}} J_{\beta} = -\frac{1}{4 K^2} \int \int_D \left[ 1- \tanh[2]( -\frac{f}{K}) \right] \left[\tanh(-\frac{f}{K})\pdv{f}{q_{\alpha}}\pdv{f}{q_{\beta}} + K \frac{\partial^2 f}{ \partial q_{\alpha} \partial q_{\beta}}\right]  dx dy
\end{equation}
Luckily, the Hessian is built into MATLAB's symbolic toolbox, so we can just call \codeword{matlabFunction(H,"File","ellipseHessian")} and the function to compute
the Hessian is saved into our working directory. Note that the function \codeword{ellipseHessian} computes a $5 \times 5 \times N_{\textrm{grid}} \times N_{\textrm{grid}}$ matrix. 
We call it for each cell $j \in \{1, \ldots, N\}$ but recall that for a given value of $\vb{q}$, the Hessian still has $(x,y)$ as free field variables. In order to integrate
the field data, we use the cell-wise Jacobian field $J_{\beta}^j (x,y)$ and the cell-wise Hessian field $H_{\alpha, \beta}^j (x,y)$ to calculate the overall Hessian field for $f$
\begin{equation*}
    \frac{\partial^2 f}{ \partial q_{\alpha} \partial q_{\beta}} = \left(\frac{1}{k}\right) \left[ \frac{\left(\sum_{j=1}^N E_j\right) \left( \sum_{j=1}^N E_j (k H_{\alpha, \beta}^j + J_{\alpha}^j J_{\beta}^j)\right) - \left(\sum_{j=1}^N E_j J_\alpha^j \right) \left(\sum_{j=1}^N E_j J_\beta^j \right)}{\left(\sum_{j=1}^N E_j\right)^2} \right],
\end{equation*}
where $E_j = e^{f_j/k}$ is used for short. To actually compute the overall integrals for $\pdv{C}{q_{\beta}}$ and $\frac{\partial^2 C}{ \partial q_\alpha \partial q_\beta }$
we use MATLAB's \codeword{trapz} function the following way
\begin{lstlisting}[style=Matlab-editor]
    trapz(y_lin,trapz(x_lin,integrandJacobian,3),2);
\end{lstlisting}
or, in the case of the Hessian,
\begin{lstlisting}[style=Matlab-editor]
    trapz(y_lin,trapz(x_lin,integrandHessian,4), 3);
\end{lstlisting}
where the integrand in the Jacobian cas is given by
\begin{equation*}
    F_{\beta}(x,y) = \left(\frac{T^2-1}{2K} \right)\frac{ \sum_{j=1}^N J_\beta^j E_j}{ \sum_{j=1}^N E_j},
\end{equation*}
where $T = \tanh( -\frac{f(x,y,\vb{q})}{K})$ and, in the case of the Hessian
\begin{equation*}
    G_{\alpha, \beta}(x,y) = 
\left(\frac{T^2-1}{4K^2} \right) \left(T \left(\frac{ \sum_{j=1}^N J_\alpha^j E_j}{ \sum_{j=1}^N E_j}\right) \left(\frac{ \sum_{j=1}^N J_\beta^j E_j}{ \sum_{j=1}^N E_j}\right) + K \frac{\partial^2 f}{ \partial q_{\alpha} \partial q_{\beta}} \right),
\end{equation*}
which, after some serious simplification becomes
\begin{equation*}
    G_{\alpha, \beta}(x,y) = \frac{T^2-1}{4Kk \left(\sum_{j=1}^N E_j\right)^2} \sum_{n=1}^N \sum_{m=1}^N E_n E_m \bigg[ \left( \frac{kT-K}{K}\right) J_\alpha^n J_\beta^m +J_\alpha^m J_\beta^m +k H_{\alpha, \beta}^m \bigg]
\end{equation*}
To make things neat we replace $B = \frac{T^2-1}{4K^2}$, $\hat{E} = \sum_{j=1}^N E_j$ and we subsitute the ratio of the smoothstep and smoothmax parameters
$S = K/k$ to attain,
\begin{equation*}
    G_{\alpha, \beta}(x,y) = \frac{B}{\hat{E}^2} \sum_{n=1}^N \sum_{m=1}^N E_n E_m \bigg[   S(k H_{\alpha, \beta}^m +J_\alpha^m J_\beta^m )+ (T-S)J_\alpha^n J_\beta^m\bigg],
\end{equation*}
Thus we can write 
\begin{equation*}
    J_\beta =  \int \int_D F_{\beta}(x,y) dx dy,
\end{equation*}
\begin{equation*}
    \dot{J}_\beta =\sum_{\alpha=1}^{5N}\dot{q}_\alpha \int \int_D G_{\alpha, \beta}(x,y) dx dy,
\end{equation*}
Now we subsitute to obtain the final integral formulae:
\begin{equation*}
    J_\beta =  \int \int_D \frac{B}{\hat{E}S} \sum_{j=1}^N J_\beta^j E_j dx dy,
\end{equation*}
\begin{equation*}
    \dot{J}_\beta = \int \int_D \frac{B}{\hat{E}^2}  \sum_{\alpha=1}^{5N}  \sum_{n=1}^N \sum_{m=1}^N \dot{q}_\alpha E_n E_m \bigg[   S(k H_{\alpha, \beta}^m +J_\alpha^m J_\beta^m )+ (T-S)J_\alpha^n J_\beta^m\bigg] dx dy.
\end{equation*}
We set out to calculate the following scalar $A = JWJ^t$ where $W = M^{-1}$ is the inverse mass matrix of our dynmical system. Note, that since we have 
only one constraint our goal is to solve for one Lagrange multiplier $\lambda$ which is given as 
\begin{equation*}
A \lambda = - \sum_{\beta =1 }^{5N }\dot{J}_\beta \dot{q}_\beta - JWQ -\kappa_s C -\kappa_d \dot{C},
\end{equation*}
Recall $C = \frac{1}{2}\int \int_D (1+T) dx dy$ and $\dot{C} = \sum_{\beta=1}^{5N} \dot{q}_\beta J_\beta$. This means we can write out our equation for $\lambda$ explicitely
\begin{equation*}
    \left( \sum_{\alpha=1}^{5N} \sum_{\beta=1}^{5N} J_\alpha W_{\alpha, \beta} J_\beta \right) \lambda = - \sum_{\beta =1 }^{5N }\dot{J}_\beta \dot{q}_\beta - \sum_{\alpha =1}^{5N} \sum_{\beta =1}^{5N}J_\alpha W_{\alpha,\beta} Q_\beta -\kappa_s \frac{1}{2}\int \int_D (1+T) dx dy -\kappa_d \sum_{\beta=1}^{5N} \dot{q}_\beta J_\beta,
\end{equation*}
Note that its convenient from a computatational point of view to bring the quadriple sum into the integral and then evaluate this using \codeword{trapz}:
\begin{equation*}
    \int \int_D \frac{B}{\hat{E}^2}  \sum_{\alpha=1}^{5N} \sum_{\beta=1}^{5N}  \sum_{n=1}^N \sum_{m=1}^N (\dot{q}_\alpha \dot{q}_\beta E_n E_m )\bigg[   S(k H_{\alpha, \beta}^m +J_\alpha^m J_\beta^m )+ (T-S)J_\alpha^n J_\beta^m\bigg] dx dy.
\end{equation*}
where the term $\dot{q}_\alpha \dot{q}_\beta E_n E_m$ is a $(5N) \times (5N) \times N \times N \times N_{\textrm{grid}} \times N_{\textrm{grid}}$ array.
Now, of course the integral we are after is the following
\begin{equation*}
    I_{\alpha, \beta}^{l,m,n} = \int \int_D \frac{B_l}{E_l^2} (E_n E_m )\bigg[   S(k H_{\alpha, \beta}^m +J_\alpha^m J_\beta^m )+ (T_l-S)J_\alpha^n J_\beta^m\bigg] dx dy.
\end{equation*}
Observing this formula, we can note that $\hat{E}$ has been replaced by $E_l$. This is actually an improvement since $\hat{E}$ was only ever part of the
smoothmax approximation, we replace it by an arbitrary $E_l$ which must be picked prior to evaluating the integral based on the maximum. I have subscripted $B_l$ and $T_l$
similarly as they depend on $E_l$. The main reason for this was due to difficulty in evaluating the integral over a reciprical sum when $N$ was arbitrary. Note that 
if we evaluate this integral symbolically in preprocessing we can essentially divide the amount of computational work by $10^6$ for a $1000 \times 1000$ grid. Even
now, the compute time scales as $N^4$ for $N$ cells. This is still intractable. A significant optimisation can be made when we realise that, from the point of view of a partial
derivative, a cell's SDF is not affected by the coordinates of another cell. Another way to put this is that the Jacobian and Hessian for a given cell $j$
depend only on the attributes of that cell and all the other partial derivatives will vanish. This actually reduces the ammount to compute by a factor of $N^2$ because
we only need to sum over the number of attributes per cell ($N_{\textrm{attrib}} = 5$ in our case) squared. A modern laptop can perform floating point
operations in the order of tens of GFLOPS. Supposing the calculation of $I_{\alpha, \beta}^{l,m,n}$ requires $1000$ floating point operations per $(m,n)$ pair. If we try
to push to $N = 1000$ yeast cells, then we will be looking at a second compute time per time step assuming the PC used operates at $1$ GFLOP. This is fine
for small $N$ but the problem doesn't scale well. In any case, with the current optimisations, we have a tractable simulaton.
\\
It is worth noting that further optimisations can be made if we employ a spatial hash map or a AABB filter which avoids computing matrix elements 
between cells which are not even nearby each other. For example, if cells $3$ and $22$ are further than $d$ apart where $d$ is the pair's maximum cell diameter
then we can set the matrix element $I_{\alpha, \beta}^{l,3,22} =I_{\alpha, \beta}^{l,22,3} = 0$ straight away (Note sure about this). We carry the full computation 
in the test phase and add the filter optimisation later.
\\
\\