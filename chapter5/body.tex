\chapter{Discussion and limitations}
\section{Evaluation of the model}
The computational model \textbf{CellColonySimulator} incorporates
design decisions based on a number of concerns. These include
\begin{itemize}
    \item The requirement for manageable run times on a gaming laptop.
    \item Modelling both colony scale and cell scale morphology.
    \item Capturing the process of mitosis in a natural but simple way.
    \item Ensuring the model is amenable to statisitical analysis.
    \item Providing a model ready for fitting to experimental micrographs of cell colonies.
    \item Crystallising the model into a form that could be analysed in a conceptually 
          rigorous setting, say within the context of functional analysis,
          bifurcation theory or algebraic topology.
\end{itemize}
Whilst the computational model and conceptual apparatus developed here take on these concerns,
this section will explore to what extent these concerns were \textit{not} met. In particualar, we comment on the
fundamental limitations of the model. In subsection \ref{conceptLims}, the commentary begins 
with an evaluation of the model regarding the primary conceptual goals.

\subsection{Conceptual limitations} \label{conceptLims}
The model was able to deploy the concept of algebraic curves with time dependent coefficients 
to represent the cells. Indeed the implicit equation for a translated, rotated, and scaled ellispe 
is a member of the ring of polynomials in two variables $\mathbb{R}[x,y]$ where coefficients are
dependent on the underlying discrete network of nodes. In a previous version of the model, 
the underlying network of nodes were initially collocated and via the elimination 
of constraints the network was able to unfold over time. This was implemented with
logical ``mask" matrices in MATLAB, but 
brought practical problems. One such problem was that branches 
could easily deplete their store of nodes, and indeed the number of nodes
chosen at the beginning of the simulation had to be arbitrary. This meant that
the previous model had prohibitive limits on the number of branches and the size
of the branches. Another limitation, was that the ODE solver was slowed
down by the need to simulate the evolution of collocated vertices.
\\

\textbf{CellColonySimulator} overcomes this issue by simply 
adding new nodes to the network, just very close by ($\delta = 0.01 L_0$).
As it turned out, this was not a consequential fault,
since the distance $\delta$ can be made so small that 
the daughter cell appears to bud from the parent cell, as 
would be observed in the case of pseduohyphal growth of baker's yeast.
What I argue is that this is in fact a conceptual fault present in the model
because it represents an arbitrary addition of a new variable,
the budding angle $\theta_{b}$ not to mention the constant (but small) budding radius 
$\delta$. What I intended  
was a ``derivation'' of cells division within a broader model,
as opposed to an imposition of more assumptions and parameters
over time to achieve the mitosis.
\\

Unsatisfyingly, the budding angle is chosen to be uniformly distributed on $[0, 2 \pi]$
and is the only source of randomness in the model. Of course the budding angle
could be chosen to be in line with the nutrient gradient vector and this would 
eliminate randomness from the model. Future work could seek to 
reintroduce randomness in the form of a stochastic process, 
such a Langevin term in the the ODEs for the node positions (and even in the nutrient PDE).
\\

A model which I have prototyped, is one wherein the nutrient field is 
effectively ``converted into nodes''. That is, the nutrient field is not simply
a scalar field of ``stuff'', but actually becomes a field of structure. The simplest
version of this is to set the nutrient field as a collection of Brownian walkers that 
can be taken up by a colony and added to the colony's network. In some sense this 
would solve the problem, since no new nodes are ever added into the simulation,
its just that the initially unstructured nodes take on a new behaviour within the cell network.
\\

Another conceptual direction is that of taking seriously the notion of 
``cell as structured randomness''. Biochemistry tells us that
cells are reducible to chemical reaction networks (CRNs), but of course to combine 
CRNs into the workings of a single cell is not trivial. This means effective
models are required, which leave out most of the labrythine details of the cell's
internal workings (such as proteins, DNA, RNA, mitotic spindles, golgi apparatus) but allow
for efficient computation. Even CRNs are limited by the fact that 
they usually do not consider the shape of a protein which plays 
a key role in its functionality. The model presented here brushes
over all of these details in favour of an effective mathematical approach, 
but future work could focus linking 
nanoscopic molecular scale and microscopic cellular scale.
\\

Regarding the shape of individual cells, 
the ellipse with variable aspect ratio is chosen to represent eukaryotic
fungal cells such as S. cerevisiae. Beyond changing 
the cell aspect ratio, there is no mechanism in \
\textbf{CellColonySimulator}
to opt for plant like cell shapes (for example) which are more blocky.
Algebraic modelling of the cell membrane (in the case of plant cells) or
a modelling other cell wall shapes (fungal or even mammal cells)
is implied here however. For instance, by chosing more elaborate
polynomials, one can derive a taxonomy of cell shapes at the cost of 
implementation simplicity and perhaps computational performance. 
An avenue to explore is seeing whether bifurcations
can be achieved in a different ring of continuous functions of $x$ and $y$.
Indeed, can the whole colony be represented by (the level section of) one function at each time,
such that the growth of the colony can be given by a homotopy 
between two functions?
\\

Philosophically, the model has some limitations. In reality, we know the growth
of fungal species such as Baker's yeast, is strongly dependent on factors such as temperature
and the environmental conditions that the colony is subjected to. This means 
that the dynamics of cell growth (the collision and interaction as well as the morphology) in 
any large colony system cannot be intrinsically defined. To put it succinctly, the dynamics of 
the colony is strongly coupled to the local environmental conditions.
\\

Does this mean that we can only model biological systems completely if all the conditions 
are incorporated? We aim to consider the simplest viable model. This, in the case of the current work, 
is a model that has two components: a cell colony as discussed in the preceeding section, and 
a nutrient field. Indeed, their coupling determines the dynamics. 
\\

This model, in which the colony appears as the ``unfolding" of a network of nodes
follows on somewhat naturally from the theory of $L$-systems, in which the underlying 
discrete structure of a system is allowed to evolve deterministically based on rules
analogous to cellular automata. A criticism of $L$-systems and other deterministic models
of plants for example, is what I call their intrinsic nature. That is, they effectively
rely on the assumption that a biological system's rules for development are somehow contained
within the structure (for instance, DNA). It is more 
scientifically accurate to think of a biological system as part of a complex network 
of other organisms and environmental conditions which determine its morphology.
\\

Physically, there are again more directions to follow up. Consider a minimalistic 
example of modelling the motion of a spherical ball rolling around on a table.
Two spatial paramters $x$ and $y$, together as a tuple are enough to say where it is. However, 
if the ball unexpectately falls from the table, then you would suddenly require
another parameter to describe the state of the system. The value of this classical example
is to demonstrate for $N$ particles and $M$ constraints that may suddenly change, 
we can derive rich and unexpected dynamics.
\\

Regarding the possibilities of geometric modelling, the best we can hope for is that 
our parametrisation is somehow dense in the space
of all possible cell colony configurations that we see experimentally. There are even
some exotic cases of cellular sytems in which a single cell can have nontrivial topologies, 
see dictyostelid cellular slime molds, \cite{glockner2016multicellularity}. Therefore, future work 
should focus on representations of dynamic cell geometries that are as free of assumptions as
is conceivable. Once an abstract enough representation of a biological system and its environment
becomes available in the theory, one can use it to study and predict novel morphologies that appear 
in nature. Hopefully it is clear that the present work is just a suggestion of the numerous possibilities
in the area of biolgical morphology viewed in the framework of ``growing geometry''.

\subsection{Dynamical limitations}
A (fixed node count) cell colony undergoes the dynamics of an overdamped spring network capable
of internal collisions driven by
an external nutrient field. The key concern is whether this dynamical structure is grounded in 
biological evidence, and whether the model is just metaphorical. Firstly, 
it is worth noting that because we are in the overdamped regime,
there are no intrinsic oscillations between connected nodes. This is because the node velocity 
is proportional to the force as opposed to the case 
of an underdamped harmonic oscillator, where the acceleration is proportial to force.
In answer to the concern posed, the current model is an effective one however
it is feasible that the model can be fitted to experimental micrograph data, 
which is commented on in section \ref{expFitting}, and thus the seven input parameters 
could be found from real data.
\\

The springs have a neutral position which is a cell length distant from the parent node.
In other words, the spring force encodes a passage between equilibrium positions of the colony.
The presence of an (external) chemotactic force coupled into the dynamics complicates 
this by gradually moving the equilibrium position for a fixed node count colony outwards.
One observation of the simulation results, was that internal cells grew beyond the nominal cell length
due to the additional force of chemotaxis. In practice,
this issue is alleviated by increasing the cell elasticity, $\lambda_2$, but this 
is nonetheless a limitation.
\\

New nodes are added to the network nearby old nodes as stated in Chapter 2. The timing
of the addition of new nodes is somewhat adhoc and provisional, but could in theory
be elevated to something more sophisticated. Starting with the colony growth rate 
$\mu(t)$ which is taken to be a fraction of the ideal growth rate $\mu^* = 0.4$ hour$^{-1}$,
cells are added in at discrete intervals of time, given by 
\begin{equation*}
    \Delta n = \bigg\lceil \frac{1}{\Delta t} \bigg\rceil.
\end{equation*}
The value $\Delta n$ is the number of time steps between two (global) mitosis events. For a
time step of $\Delta t = 0.02$, this amounts to $\Delta n = 50$ time steps. The amount of 
nodes added at this time step, denoted $n^*$ (a multiple of $\Delta n$), is given by 
$(e^{\mu(t_{n^*})} -1) N_{\textrm{nodes}}(t_{n^*})$ where $\mu (t_{n^*})$ 
is the average value of the nutrient field (recall that the nutrient field is assumed to 
be $1.0$ at the simulation outset). The assumption that growth rate is defined
in terms of the nutrient field makes sense, however the assumption of it being \textit{globally} 
defined thus is somewhat dubious. A possible impovement is to draw up a coarse 
grid in which each square can fit a dozen or more cells, and say that 
the growth rate $\mu = \mu(x,y,t)$ is locally defined by averaging nutrient per grid square. 
Of course, the reason 
that a spatially independent growth rate, $\mu = \mu(t)$, was chosen was for 
implementation simplicity.
\\

The periodic structure of adding cells in at simultaneous snapshots of time,
is an artificial choice based on implementation simplicity alone. 
A higher fidelty model could be constructed in which cells bud off at any time.
One way to do this is to calibrate mitosis times per cell based on cell cycle data
and allow a cell to undergo mitosis if a certain threshold of nutrient is locally
available. This would not be difficult to implement, but 
comes with a few conceptual caveats. One issue is that 
a (discrete) logic would need to be introduced per cell,
which opens up a fascinating area of study (``cell as computer program''), however 
this is outside the scope of this work.
\\

Another limitation of the dynamical model is overly simplistic collisions.
The collision mechanism is node based, as 
opposed to edge based or even SDF based.
An attempt to improve upon the expedient hard ball collision scheme, 
is developed in Appendix \ref{collisionModel}. Therein,
constrained dynamics techniques are employed to ensure no pairwise
overlap of the biomass level sections. It is clear that constrained
dynamics is the way to go, but my unique implementation is 
not computationally efficient and therefore was not
used in \textbf{CellColonySimulator}.
\\

Finally, the use of a reaction-diffusion PDE is traditional in mathematical biology, however
in light of the insights developed here, future work could focus on modelling 
random fluctuations in nutrient concentration, or even introducing multiple interacting 
morphogens such as in Turing's morphogenesis paper \cite{turing1990chemical}.



\subsection{Computational consolations and cell death}

\textit{``Now it is autumn and the falling fruit
and the long journey towards oblivion....
Have you built your ship of death, O have you?
O build your ship of death, for you will need it."}
\\
An excerpt from \textit{The Ship of Death}, by D.H.Lawrence.
\\

I had originally written a MATLAB script called
\textit{removeCell.m} but I decided not to include this 
in \codeword{CellColonySimulator}. The reasons 
for apoptosis (the triggering of a cell death event)
are probably very complex with respect to the whole organism.
That being said it should be possible 
to analyse using statistical methods whether 
the cause is within the cell or within the whole 
colony, or both.
\\

As far as computational efficiency and the choice of programming language is concerned, 
it was my initial intent to code \textbf{CellColonySimulator} in the C programming language 
using the structs of arrays (SOAs) data orientated approach. 
\\

In this direction, significant speedups can be made if skilled use of 
parallel threading and GPU compute (through CUDA) are employed. Visualisation 
of the colony could be achieved by converting the colony's
implicit representation into a surface mesh and rendered on the GPU
via a vertex shader program in OpenGL. Another alternative 
is to use ray marching where the object to be rendered is projected 
onto a quadrilateral arranged in the viewport. Rays
projected forward from this quadrilateral locate the object via 
the object's signed distance field and lighting effects can
be easily integrated. Because 
this can be done in a fragment shader in OpenGL it 
would provide a quick visualisation without much 
coding involved. The meshed solution, mentioned first,
despite being more clunky, seems more flexible,
because it can be integrated easily with other meshed-based
scene elements, such as plot axes and labels.
\\

Another approach, is to simulate the colony dynamics in a lower 
level language like C or Fortran, saving the node positions and nutrient 
field at each moment. The biomass can then be reconstructed
from the node positions in a language like MATLAB or Python,
where post-processing such as plotting and statistics can be done.
The best of both words would be to toggle on or off 
OpenGL visualisation in C, to check a
preliminary visualisation for any obvious issues, whilst saving 
the run data for post-processing in another language.





\section{Readiness for experimental validation and fitting}\label{expFitting}
Observing real yeast colonies, we note that multiple
seperate budding networks form. That is to say,
the present model would need to be further modified 
to include detachment events where new cells form a 
seperate disconnected colony.
\\

The addition of robust cell death and cell detachment events would 
prime the model to be experimentally validated. This 
would likely introduce new parameters $\lambda$ and more 
overall complexity, so an in-depth, flexible and robust modelling platform 
in C (as mentioned earlier) integrated with a good data analytics package 
would be ideal to carry out comparisions with experimental yeast micrographs
at a managable run time. The theory is that the highly optimised run 
script can be called upon from MATLAB or Python when required.
\\

Statistical best fits to experimental morphological
data could be derived by using metrics like compactness and
average radius. The question becomes: \textit{for 
what model parameter values can one minimise the difference squared
between the simulated (ensemble average) and experimental images?}
One could use a feed foward 
neural network (NN) in order to learn the model itself, 
modelling the output parameters, compactness and others, as a function of the input 
variables, $\lambda$.











