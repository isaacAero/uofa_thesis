\conclusions
The components of this thesis when considered individually
are adapted from previous mathematical 
modelling, sometimes in biology, sometimes not. The value of 
what has been done emerges from the interplay of the elements
\begin{enumerate}
    \item A PDE nutrient field,
    \item An overdamped and growing spring network,
    \item A cell colony given as a level-section of a polynomial,
\end{enumerate}
all of which I feel were necessary for the model's overall 
synthesis to occur. I by no means believe 
that even a small proportion of the total causes and effects
have been accounted for in regard to baker's yeast proliferation 
in nutrient-poor conditions. But it is my hope 
that new model \textit{triple-points} or even \textit{multi-stable symbolic states} 
can be sought after in future 
studies.
\\

Qualitatively different types of models, when coupled
in this attentive way can give rise to new dynamics not found in 
any singular model. That being said, 
the process of learning and tending to more models than one can be
laborious, if not grounded in some sort of overarching 
framework, even if that framework is to disappear 
when the total project emerges.
\\

\textit{A word on my own working process.}
The openness to testing different models without 
the promise of intellectual refuge in one prior form, came 
from my conviction that \textit{life itself} is irreducible
and heterogenous. Our mathematical models sometimes reflect 
this truth (that I have come to understand through struggle, 
hurt and loneliness), and sometimes 
they do not. In that sense, this thesis is a 
\textit{personal reckoning} with the very structuring 
force of our own assumptions. But there are other 
forces at play aswell.
\\

The pressure of scientific tradition was a necessary 
influence on my project, and so was 
my own opposing reaction to its reductionist tendancies.
Without these forces this project would not have 
branched out into what it became. Since 
I too am a living being whose cells 
are in constant interaction with the environment, 
it was necessary to bring my own subjectivity
into this tango of symbols.
\\

In classical mechanics, forces are vectors that 
are summed to give the net force. What we have 
come to forget is that individual forces themselves
could have different mathematical structures associated 
to them. There may be more natural or ergonomic 
formalisms associated to each force. 
The dirrect approach then, is to consider 
``force formalisms" that allow for designed trajectories.
\\

It seems, ``change how you look at it, and you will see something 
different,'' is a dismissal of form. That is not the case. 
There is a rich history of mathematics that studies 
subjectivity directly, but you can only see that if you sit with 
it for long enough. Mathematics is wedded to form and form 
is about perception.
\\

Ironically then, for an applied mathematics thesis grounded 
in the observation of an organism like \textit{S. cerevisiae},
this work is truly endebted to what I imagine 
subjects \textit{topology}, \textit{geometry}, and 
\textit{algebra} to be like. I have studied 
these topics very haphazardly and hope to refine my 
understanding over the next few years.
